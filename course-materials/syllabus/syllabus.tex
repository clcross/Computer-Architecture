\documentclass[12pt]{article}
\textwidth=7in
\textheight=9.5in
\topmargin=-1in
\headheight=0in
\headsep=.5in
\hoffset  -.85in

\pagestyle{empty}

\renewcommand{\thefootnote}{\fnsymbol{footnote}}

\usepackage{hyperref}

\begin{document}

\begin{center}
{\bf CSCI 330\ Sec. 01  Computer Architecture \ \ TR 0930-1050 Lab 1050-1150,  Room: LC 210
}
\end{center}

\setlength{\unitlength}{1in}

\begin{picture}(6,.1) 
\put(0,0) {\line(1,0){6.25}}         
\end{picture}

\renewcommand{\arraystretch}{2}
\setlength{\tabcolsep}{6pt} % General space between cols (6pt standard)
\renewcommand{\arraystretch}{.5} % General space between rows (1 standard)

\vskip.15in
\noindent\textbf{Instructor:} Dr. Paul E. West, Ashby Hall 206, Phone: (843)-863-7329, Email: pwest@csuniv.edu

\vskip.15in
\noindent\textbf{Office Hours:} M 1100-1200, TR 0740-0920, TR 1200-1350, and by appointment.

\vskip.15in
\noindent\textbf{Textbooks:} \\
\underline{Computer Organization and Design}, David A. Patterson and John L. Hennessy, ISBN: 0124077269. \\

\vskip.15in
\noindent\textbf{Github:} \\
I drive most of my classes from github.  Assignments and lectures are located on Github here: \url{https://github.com/csu-cs/csci-330-spring-2020}.  If you are not able to view, please send me an email so I can grant you access.\\
\textbf{Note:} The content for this course in on Github!

\vskip.15in
\noindent\textbf{Course Description:}
This course explores the interdependencies among assembly language, computer organization and design with a focus on the concepts that are the basis for current computer technology. Stored-program concept, computer arithmetic, datapath and control, microprogramming, logic design, truth tables, logic gates, programmable logic arrays, control, pipelining, the memory hierarchy, and caches.

\vskip.15in
\noindent\textbf{Introduction:}
This course studies general principles and concepts of computer systems.  First, you will learn the theoretical concepts of various kinds of computer components. Second, you will learn how to utilize these components to improve program performance.  This course will cover:

\begin{itemize}
\item Boolean Algebra
\item Digital Gates
\item Logic Circuit
\item The fundamentals computer components.
\item Processor Architecture.
\item Optimizing Program Performance.
\item The memory Hierarchy.
\item Measuring Program Execution Time.
\item High Performance Computing.
\item Improve student written and verbal skills.
\item Increase student knowledge of ethical issues.
\end{itemize}

\vskip.15in
\noindent\textbf{Course Objectives/Learning Outcomes:}
ABET Student Outcomes: The following student outcomes shall be supported by this coursework:
\begin{enumerate}
%\item An ability to apply knowledge of computing and mathematics appropriate to the discipline.
\item An ability to analyze a problem, and identify and define the computing requirements appropriate to its solution.
\item An ability to design, implement, and evaluate a computer-based system, process, component, or program to meet desired needs.
\item An ability to communicate effectively with a range of audiences.
\item Recognize professional responsibilities and make informed     judgments in computing practice based on legal and ethical principles. 
\item Function effectively as a member or leader of a team engaged in activities appropriate to the program’s discipline. 
\item Apply computer science theory and software development fundamentals to produce computing-based solutions. 
%\item An ability to use current techniques, skills, and tools necessary for computing practice.
%\item An ability to apply mathematical foundations, algorithmic principles, and computer science theory in the modeling and design of computer-based systems in a way that demonstrates comprehension of the tradeoffs involved in design choices.
%\item An ability to apply design and development principles in the construction of software systems of varying complexity.
\end{enumerate}


\vskip.15in
\noindent\textbf{Teamwork:} There is an expectation of teamwork in many of the class/lab projects. The professor will use his/her discretion as to the team membership and will direct teams to produce a single solution among the teammates. Teamwork is a highly valued skill in the workplace and society as a whole. Through these teamwork exercises the goal is to develop an understanding of what makes teams successful and to be able to function effectively as a teammate.

\vskip.15in
\noindent\textbf{Attendance and Late Work:}  Since work will be handled through ScrumDo, the due dates are through the 4 iterations in Scrumdo.  Please see Scrumdo for more details.

\vskip.15in
\noindent\textbf{Midterm Grade}: \\ \\
Since I have to turn in midterm grades, I will tag the cards that count toward your midterm.

\vskip.15in
\noindent\textbf{Grading}: \\ \\
Your grade is the sum of points earned divided by the amount assigned multiplied by 100 to get a percentage.

\vspace*{.15in}
\noindent\textbf{Grade Scale:} \\ \\
\begin{tabular}{|l|l|}
\hline
100 - 90 & A \\ \hline
89 - 87 & B+ \\ \hline
86 - 80 & B \\ \hline
79 - 77 & C+ \\ \hline
76 - 70 & C \\ \hline
69 - 60 & D \\ \hline
below 60 & F \\ \hline
\end{tabular}

\vspace*{.15in}
\noindent\textbf{Tentative Schedule:} \\ \\
\begin{tabular}{|l|l|}
\hline
Week & Topic \\ \hline
1    & Overview \\ \hline
2    & Implementation Technologies \\ \hline
3    & Latches and Flip-Flops \\ \hline
4    & Standard Combinational Components \\ \hline
5    & Computer Arithmetic \\ \hline
6    & Circuit Optimization \\ \hline
7-8  & The Processor (Chapter 4) \\ \hline
9    & Spring break \\ \hline
11   & Assembly \\ \hline
11-12& Memory (Chapter 5) \\ \hline
13   & Parallel Processing (Chapter 6) \\ \hline
14   & Misc Topic \\ \hline
15   & Review/Presentations \\ \hline
\end{tabular}

\vskip.15in
\noindent\textbf{Student Representatives}:
These are students who are designated by letter to represent the University on official business, e.g., athletic, music, and similar events. If officially scheduled absences cause these students to miss tests, assignments, and/or other similar academic activities, University policy allows these to be made up without penalty. In accordance with this policy, Student Representatives may opt to either make up tests prior to departure, or supplanting missed tests with the final exam grade. Final exams must always be taken prior to departure to avoid an Incomplete for the course. Scheduled assignments remain subject to the lateness policy and must be turned in before departure to avoid lateness penalties. Student Representatives are responsible to inform the instructor of official absences and to make all appropriate arrangements.

\vskip.15in
\noindent\textbf{Extra Help}:  Dot not hesitate to come to my office during office hours or by appointment to discuss a homework problem or any aspect of the course.

\vskip.15in
\noindent\textbf{Disability Services}: 
If there is any student in this class who thinks they may have need
of accommodations, they should review the requirements/procedures on Disability
Services’ website http://www.csuniv.edu/student-success/disabilityservices.php. Once a
student has been approved to receive accommodations through Disability Services,
they will need to contact this instructor.
Nondiscrimination Policy and Student Rights
Charleston Southern University does not illegally discriminate on the basis of race,
color, national or ethnic origin, sex, disability, age, religion, genetic information, veteran
or military status, or any other basis. Inquiries regarding the non-discrimination policies
should be directed to Latitia R. Adams, Title IX Coordinator, 843-863-7374,
ladams@csuniv.edu. Students should refer to the CSU Student Handbook
(http://www.csuniv.edu/docs/studenthandbook.pdf) to be fully informed of their rights
and remedies.
Evaluations: In order to pursue our mission of ‘Academic Excellence in a Christian
Environment’, it is important that we receive feedback from students to let us know how
are doing. In order to save time and paper this process is online, and should be
available sometime in the second half of the semester. Students are strongly
encouraged to complete the short evaluation, which is entirely anonymous. Your
professor will let you know when this is active, and you can then access it through your
MyCSU account. We greatly value your opinion!

\vskip.15in
\noindent\textbf{Academic Integrity}: 
As a liberal arts university committed to the Christian faith, Charleston Southern
University seeks to develop ethical men and women of disciplined, creative minds and
lives that focus on leadership, service and learning. The Honor System of Charleston
Southern University is designed to provide an academic community of trust in which
students can enjoy the opportunity to grow both intellectually and personally. For these
purposes, the following rules and guidelines will be applied.
"Academic Dishonesty" is the transfer, receipt, or use of academic information, or the
attempted transfer, receipt, or use of academic information in a manner not authorized
by the instructor or by university rules. It includes, but is not limited to, cheating and
plagiarism as well as aiding or encouraging another to commit academic dishonesty.
"Cheating" is defined as wrongfully giving, taking, or presenting any information or
material borrowed from another source - including the Internet by a student with the
intent of aiding himself or another on academic work. This includes, but is not limited to
a test, examination, presentation, experiment or any written assignment, which is
considered in any way in the determination of the final grade.
"Plagiarism" is the taking or attempted taking of an idea, a writing, a graphic, music
composition, art or datum of another without giving proper credit and presenting or 
attempting to present it as one's own. It is also taking written materials of one's own that
have been used for a previous course assignment and using it without reference to it in
its original form.
Students are encouraged to ask their instructor(s) for clarification regarding their
academic dishonesty standards. 

\vskip.15in
\noindent\textbf{Course Evaluations}:
CSU collects course evaluations via the web. Instructions on how to access this system and how to evaluate the course will be available midway through the class. I encourage you to take this seriously and provide constructive feedback for improving the class. 

\vskip.15in
\noindent\textbf{Important Dates}:
\begin{center} \begin{minipage}{5in}
\begin{flushleft}
Drop/Add Deadline \dotfill January 17\\
End of regular work \dotfill March 20 \\
End of final work \dotfill TBD \\
\end{flushleft}
\end{minipage}
\end{center}

\end{document}
