%Paul E. West

%\documentclass[xcolor=svgnames]{beamer}
\documentclass{beamer}
\usepackage[boxed,vlined,figure]{algorithm2e}

%\usecolortheme[named=FireBrick]{structure}
%\usecolortheme[named=black]{structure}
%\usecolortheme{beetle}
%\usecolortheme{beaver}
%\usecolortheme{crane}
%\usecolortheme{dolphin}
%\usecolortheme{dove}
%\usecolortheme{fly}
%\usecolortheme{lily}
\usecolortheme{orchid}
%\usecolortheme{rose}
%\setbeamercolor{background canvas}{bg=Gold!25}
%\setbeamercolor{background canvas}{bg=Black!100}
%\setbeamercolor{foreground}{bg=Gold!25}
%\setbeamercolor{normal text}{fg=green,bg=black}
%\setbeamercolor*{palette primary}{use=structure,fg=green,bg=black}

\mode<presentation>{
    \usetheme{Darmstadt}
    \setbeamercovered{invisible}
    %\setbeamercovered{transparent}
    \setbeamercolor*{palette primary}{use=structure,fg=white,bg=blue}
    \setbeamercolor*{palette secondary}{use=structure,fg=white,bg=blue}
    \setbeamercolor*{palette tertiary}{use=structure,fg=white,bg=blue}
}

\usepackage[english]{babel}
\usepackage[latin1]{inputenc}
\usepackage{times}
\usepackage[T1]{fontenc}
%\usepackage{epsfig}
\usepackage{ulem}
\usepackage{color,soul}

\usepackage{graphicx}
\usepackage{amssymb}
\usepackage{url,hyperref}
\definecolor{beamer@blendedblue}{rgb}{1,.6,.2}
%\usepackage{tikz}
%\usetikzlibrary{shapes}
%\usetikzlibrary{arrows}
%\tikzstyle{block}=[draw opacity=0.7, line width=1.4cm]
\usepackage{listings}
\lstset{language=C}
\lstset{tabsize=4}
\lstset{basicstyle=\tiny}


%\usecolortheme[overlystylish]{albatross}
%\usecolortheme[]{lily}
%\usecolortheme[]{albatross}
%\usecolortheme[]{orchid}
%\setbeamercolor{normal text}{fg=green!10}

\title{CSCI 330: Computer Architecture}
\author{Dr. Paul E. West}

\institute{
  Department of Computer Science\\
  Charleston Southern University
}

\date{January 13, 2020}

\subject{Computer Architecture}
%\keywords{Performance Counters, Multicore}

%\pgfdeclareimage[height=1.0cm]{university-logo}{../imgs/csu-logo}
\pgfdeclareimage[height=0.75cm]{university-logo}{../imgs/csu-logo}
%\pgfdeclareimage[height=0.50cm]{university-logo}{../imgs/csu-logo}
\logo{\pgfuseimage{university-logo}}

\begin{document}

\begin{frame}
  \titlepage
\end{frame}

\section{Overview}
\subsection{}

\begin{frame}{Complex Yet Simple}
\begin{itemize}
\item Computers are some of the more complex devices humans have created.
\item Yet, are simple in many ways.  EX:
\begin{itemize}
\item A CPU (where a lot of the ``work'' occurs) is basically a bunch of transistors etched on silicon.
\item by `a bunch' I mean billions
\end{itemize}
\end{itemize}
\end{frame}

\begin{frame}{From C to Execution}
\begin{itemize}
\item How does the code that you type in go from text to executing on the computer?
\item How does a piece of silicon turn you ``English-like'' C code into a dynamic computation?
\item We will explore these in more depth.
\end{itemize}
\end{frame}

\begin{frame}{Logic}
\begin{itemize}
\item The computer executes your program (set of instructions) through electronic signals.
\item These signals (in general) are executed in steps (clock cycle).
\item Our job here is to see how this is done and give you an idea of how a computer can be built to do computations.
\end{itemize}
\end{frame}

\section{State of Computers}
\subsection{}
\begin{frame}{The Computer Revolution}
\begin{itemize}
\item Progress in computer technology
\begin{itemize}
\item Underpinned by Moore's Law 
\end{itemize}
\item Makes novel applications feasible
\begin{itemize}
\item Computers in automobiles
\item Cell phones
\item Human genome project
\item World Wide Web
\item Search Engines
\end{itemize}
\item Computers are pervasive
\end{itemize}
\end{frame}

\begin{frame}{Classes of Computers}
\begin{itemize}
\item Personal computers
\begin{itemize}
\item General purpose, variety of software
\item Subject to unit cost vs performance tradeoff
\end{itemize}
\item Server computers
\begin{itemize}
\item Network based
\item High capacity, performance, reliability
\item Range from small servers to building sized
\end{itemize}
\end{itemize}
\end{frame}

\begin{frame}{Classes of Computers}
\begin{itemize}
\item Supercomputers
\begin{itemize}
\item High-end scientific and engineering calculations
\item Highest capability but represent a small fraction of the overall computer market 
\item \url{http://www.top500.org} maintains a list
\end{itemize}
\item Embedded computers
\begin{itemize}
\item Hidden as components of systems
\item Stringent power/performance/cost constraints
\end{itemize}
\end{itemize}
\end{frame}

\subsection{}
\begin{frame}{PC Mobile Tablet Console Market Share}
\includegraphics[width=1.0\textwidth]{../imgs/pc-mobile-sales.png}
\end{frame}

\begin{frame}{Mobile Devices and Cloud}
\begin{itemize}
\item Personal Mobile Device (PMD)
\begin{itemize}
\item Battery operated
\item Connects to the Internet
\item Hundreds of dollars
\item Smart phones, tablets, electronic glasses, smart watches
\end{itemize}
\item Cloud computing
\begin{itemize}
\item Warehouse Scale Computers (WSC)
\item Software as a Service (SaaS)
\item Portion of software run on a PMD and a portion run in the Cloud
\item Amazon and Google
\end{itemize}
\end{itemize}
\end{frame}

\section{Course}
\subsection{}
\begin{frame}{What you Will Learn}
\begin{itemize}
\item How programs are translated into the machine language
\item And how the hardware executes them
\item The hardware/software interface
\item What determines program performance
\item And how it can be improved
\item How hardware designers improve performance
\item What is parallel processing
\end{itemize}
\end{frame}

\begin{frame}{Patterns to Keep in Mind}
\begin{itemize}
\item The three Ps for improving performance:
\begin{itemize}
\item Parallelism
\item Pipelining
\item Prediction
\end{itemize}
\item Small is fast, Big is slow
\item You must think in parallel!
\item Verilog
\begin{itemize}
\item A different language used to describe processors
\end{itemize}
\end{itemize}
\end{frame}


\section{Administrativa}
\subsection{}
\begin{frame}{About the Professor}
\begin{itemize}
\item PhD from Florida State University in Computer Science
\item Faculty Experience:
\begin{itemize}
\item Charleston Southern: Assistant 2015-Present
\item College of Charleston: Adjunct 2013-2014
\end{itemize}
\item Work Experience:
\begin{itemize}
\item Naval Research Lab: 2018 - Present
\item Google (2014): Android Bluetooth/Wi-Fi/Telephony
\item SPAWAR (2009-2014, 2015): Communication systems
\item DenimGroup (2004-2005): Start-up; web design and network security
\end{itemize}
\end{itemize}
\end{frame}

\begin{frame}{Github}
\begin{itemize}
\item The programming parts of your assignments will be submitted through Github.
\item Please create an account on Github and email me your username.
\item More instruction to follow...
\end{itemize}
\end{frame}

\begin{frame}{ScrumDo}
\begin{itemize}
\item A board to organize our workflow for this course.
\item We will use if it will allow us to create enough users.
\item Lets find out...
\end{itemize}
\end{frame}

\begin{frame}{Syllabus}
Lets go over the syllabus...
\end{frame}

\section{Architecture Overview}
\subsection{}

\begin{frame}{Discussion}
\begin{itemize}
\item Any questions about the syllabus or the course?
\item Have you seen anything in the news?
\end{itemize}
\end{frame}

\begin{frame}{Below Your Program}
\begin{columns}
\column{0.40\textwidth}
\includegraphics[width=1.0\textwidth]{../imgs/below-your-program.png}
\column{0.60\textwidth}
\begin{itemize}
\item Application software
\begin{itemize}
\item Written in high-level language
\end{itemize}
\item System software
\begin{itemize}
\item Compiler: translates HLL code to machine code
\item Operating System: service code
\begin{itemize}
\item Handling input/output
\item Managing memory and storage
\item Scheduling tasks and sharing resources
\end{itemize}
\end{itemize}
\item Hardware
\begin{itemize}
\item Processor, memory, I/O controllers
\end{itemize}
\end{itemize}
\end{columns}
\end{frame}

\begin{frame}{Levels of Code}
\begin{columns}[c]
\column{0.60\textwidth}
\begin{itemize}
\item High-level language
\begin{itemize}
\item Level of abstraction closer to problem domain
\item Provides for productivity and portability 
\end{itemize}
\item Assembly language
\begin{itemize}
\item Textual representation of instructions
\end{itemize}
\item Hardware representation
\begin{itemize}
\item Binary digits (bits)
\item Encoded instructions and data
\end{itemize}
\end{itemize}
\column{0.40\textwidth}
\includegraphics[width=1.0\textwidth]{../imgs/levels-of-code.png}
\end{columns}
\end{frame}

\begin{frame}{Inside the Processor (CPU)}
\begin{itemize}
\item Datapath: performs operations on data
\item Control: sequences datapath, memory, ...
\item Cache memory
\begin{itemize}
\item Small fast SRAM memory for immediate access to data
\end{itemize}
\end{itemize}
\end{frame}

\begin{frame}{Intel Haswell}
\includegraphics[width=1.0\textwidth]{../imgs/coffee-lake.png}
\end{frame}

\begin{frame}{Apple A13}
\includegraphics[width=0.8\textwidth]{../imgs/a13.png}
\end{frame}

\begin{frame}{Safe Place for Data}
\begin{itemize}
\item Volatile main memory
\begin{itemize}
\item Loses instructions and data when power off
\end{itemize}
\item Non-volatile secondary memory
\begin{itemize}
\item Magnetic disk
\item Flash memory
\item Optical disk (CDROM, DVD)
\end{itemize}
\end{itemize}
\end{frame}

\begin{frame}{Networks}
\begin{itemize}
\item Communication, resource sharing, nonlocal access
\item Local area network (LAN): Ethernet
\item Wide area network (WAN): the Internet
\item Wireless network: WiFi, Bluetooth
\end{itemize}
\end{frame}

\begin{frame}{Technology Trends}
\begin{columns}[c]
\column{0.50\textwidth}
\begin{itemize}
\item Electronics technology continues to evolve
\begin{itemize}
\item Increased capacity and performance
\item Reduced cost
\end{itemize}
\end{itemize}
\column{0.50\textwidth}
\includegraphics[width=1.0\textwidth]{../imgs/tech-trends.png}
\end{columns}
\begin{tabular}{l l r}
Year & Technology & Relative performance/cost \\
1951 & Vacuum tube & 1 \\
1965 & Transistor & 35 \\
1975 & Integrated circuit (IC) & 900 \\
1995 & Very large scale IC (VLSI) & 2,400,000 \\
2013 & Ultra large scale IC & 250,000,000,000 \\
\end{tabular}
\end{frame}

\begin{frame}{Semiconductor}
\begin{itemize}
\item Silicon:  semiconductor
\item Add materials to transform properties:
\begin{itemize}
\item Conductors
\item Insulators
\item Switch
\end{itemize}
\end{itemize}
\end{frame}

\begin{frame}{Manufacturing ICs}
\includegraphics[width=1.0\textwidth]{../imgs/manufacture-ics.png}
\begin{itemize}
\item Yield: proportion of working dies per wafer
\end{itemize}
\end{frame}

\begin{frame}{Intel Core i7 Wafer}
\includegraphics[width=0.5\textwidth]{../imgs/i7-wafer.png}
\begin{itemize}
\item 300mm wafer, 280 chips, 32nm technology
\item Each chip is 20.7 x 10.5 mm
\end{itemize}
\end{frame}

\section{Verilog}
\subsection{}

\begin{frame}
\begin{itemize}
\item We will be using Verilog throughout the course
\item I \textbf{\textit{highly}} recommend Virtualizing Ubuntu and running everything from there.
\item \textbf{Homework:} Please download iVerilog (Icarus Verilog) and be ready to start coding next class
\begin{itemize}
\item Icarus Verilog: Open source/free Verilog tool chain
\item Linux: use your package manager or \url{http://iverilog.icarus.com/}
\item Windows: \url{http://bleyer.org/icarus/}
\end{itemize}
\item Verilog is a industry language used to design processors
\item We will be building a simplified x86 processor
\begin{itemize}
\item x86 = 32-bit processor, which is the root of virtually all PCs today.
\item modern day CPUs use x86\_64, a 64-bit version of x86.
\end{itemize}
\end{itemize}
\end{frame}

\end{document}
